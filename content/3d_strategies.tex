\section{3D exploration strategies}
--3D(Nearest frontier)
%In contrast the approaches for exploration in 3D terrains are relatively lesser. The notion of coverage of known terrains rather than exploration of the unknown has gained more prominence in 3D worlds. In [9] Eidenbenz presents an approach for minimum number of guards to cover a known 3D terrain. Recently Joho
%and others present a paper for single robot exploration in 3D
%worlds [10] through a metric that is analogous to . This and
%earlier approaches such as [11] are more focused on generating
%3D maps and deal with single robots operating on multiple
%planes. The degree of freedom along the height axis is almost
%zero as they are ground robots. Height variations occur in those
%approaches only due to undulations of the ground plane. Our
%survey showed no work based on 3D heightmap based multirobot exploration.


%[9] S Eidenbenz, “Approximation Algorithms for Terrain
%Guarding”, Information Processing Letters,82(2002): 99-105
%[10] Dominik Joho, Cyrill Stachniss, Patrick Pfaff, and Wolfram
%Burgard, “Autonomous Exploration for 3D Map Learning”,
%Autonome Mobile Systeme (AMS). Kaiserslautern, Germany,
%2007.
%[11] Surmann H, Nuchter A, Hertzberg J, “An autonomous
%mobile robot with a 3D laser range finder for 3D exploration and
%digitalization of indoor environments”, Robotics and
%Autonomous Systems, 45(3-4):181–198, 2003. 

\subsection{Non-coordinated strategies}