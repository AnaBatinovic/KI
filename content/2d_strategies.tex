\section{2d exploration strategies}

Exploration algorithms can be grouped into centralized, distributed and decentralized. In the  centralized approach, each mobile robot receives tasks from a single central \emph{leader} which runs the overall planning algorithm, and afterwards the mobile robot sends its info back to the leader. Centralized assignment may be less practical due to communication limits (\cite{Dias2000}), robustness issues (\cite{Dias2006}), or time required for algorithm execution and scalability (\cite{Julia2012}). An advantage of centralized approach is that optimal plans can be found (\cite{Yan2011}). For instance, Sharma et al. \cite{SharmaHonc2016} used a centralized exploration approach based on routing priority. The algorithm keeps track of the frontiers, assigning them to robots  whenever they fall into a trap situation.

In contrast to centralized approaches, in a decentralized approach, the mobile robots are completely independent throughout the exploration process. Each mobile robot has its own local knowledge of the world and can decide its future actions by taking into account its current context and tasks, its own capacities and the capacities of the other mobile robots, through a negotiation process (\cite{Yan2013}). Moreover, it typically has better reliability, flexibility, adaptability and robustness (\cite{Zlot2002}). 
 
There are several representative approaches from the both centralized and decentralized 2D exploration strategies described in the following text. 


\subsection{Nearest Frontier Approach} 

\begin{figure}[t!]
	\centering
	\fbox{\includegraphics[width=1.0\columnwidth]{./pictures/rviz_environment.pdf}}
	\caption {The environment is represented by a 2D map, with an occupancy grid that divides the map into cells: white cells describe free while grey cells unknown space. Black cells define occupied space (obstacles). Frontiers (red) and filtered frontiers (green).}
	\label{fig:environment}
\end{figure}
An unexplored area is usually represented using an occupancy grid map introduced by \cite{Moravec}. While a robot moves, an occupancy likelihood for each cell of the grid is updated with the information of sensors. Depending on this occupancy likelihood, cells can be classified as free, occupied or unknown (Fig. \ref{fig:environment}). Using an occupancy grid a mobile robot can reach an unexplored zone navigating to the frontier cells that separate the free cells from the unknown cells known as \textit{frontiers} \cite{Yamauchi1997}. A general frontier exploration diagram is shown in Fig. \ref{fig:flow_diagram}.

\begin{figure}[t!]
	\centering\includegraphics[width=0.85\columnwidth]{./pictures/flow_diagram.pdf}
	\caption{Flow diagram for the frontier exploration strategies. In case a robot does not have a goal point, goal is computed, assigned, a path to the assigned goal is obtained and a trajectory executed.}
	\label{fig:flow_diagram}
\end{figure}

Yamauchi's technique consists in selecting the shortest path to the nearest frontier. In this way, the target cell selected by this technique $t_{NF}$ is:

\begin{equation}
t_{NF} = \argmin_{a \in F} L(a), 
\label{equation:t-nf}
\end{equation}

where $L(a)$ represents the length of the shortest path to reach the cell $a$ ($a_{i}$, $a_{j}$) and $F$ the subset of the frontier cells \cite{Julia2012}. As it can be noticed, (\ref{equation:t-nf}) only takes into account the cost of reaching a frontier cell and does not provide any coordination mechanism. In case a single mobile robot system is extended to a multi-robot system, mobile robots may select the same frontier if they are situated in nearby positions. For instance, \cite{Yamauchi1998} extended his nearest frontier approach to multiple mobile robots using global maps built by each robot with the information provided by all robots. Since mobile robots share the acquired information, exploration is cooperative, but mobile robot movements are uncoordinated. When the robots are in close positions it is likely that they choose the same frontier to explore if no other coordination mechanisms are considered.

Frontier exploration strategies are also extended to a multi-robot system in \cite{Simmons2000} and \cite{Burgard2005}. Simmons \cite{Simmons2000} used a semi-distributed model where a centralized module integrated local data from a team of mobile robots. The team used a probabilistic technique to build a global map in a coordinated fashion. Due to the problem of absolute positioning techniques in indoor environments, robots must estimate their local pose in an environment, leading to odometry errors. Simmons used probability calculations to estimate the local pose of each robot, and built a global map by joining each individual robots local map. 
While Simmons \cite{Simmons2000} implements coordination between robots by sharing of map information and reducing the utility of frontier points in the vicinity of an allotted point, Burgard \cite{Burgard2005} came up with an elegant bidding process. 

A dense frontier points detection method implemented by Orsulic (\cite{Orsulic2019}) is an extension to Google Cartographer (\cite{Hess2016}) that has achieved good results in terms of wall-time per frontier update, which greatly speeds up exploration process. Orsulic used nearest frontier approach in order to explore an area. 

Rekeleitis \cite{Rekeleitis2000} covered terrains with multiple robots where at least one robot was stationary and posed as an observer. In \cite{Fox2006} a decision theoretic approach to multi-robot exploration was presented where the main problem was to decide whether a mobile robot should explore the terrain or to verify the hypothesis of other robots whose states are not mapped into a common reference frame. The nearest unexplored region technique is also used by Wullschleger \cite{Wullschleger99}, Santosh \cite{Santosh2008}, Murphy and Newman \cite{Murphy2008}, to name a few more.


\subsection{Cost-Utility Approach}

Generally, in a cost-utility approach a goal point maximizes the benefit between cost and utility. The utility is measured in terms of the expectation of the information incorporated to the occupancy map from the position of the goal point. An example of cost-utility approach was presented by González-Baños and Latombe \cite{GonzlezBaos2002}. Frontier cells are designated as candidate destinations and the benefit $B_{CU}(a)$ to reach a candidate cell a is evaluated according to the following expression \cite{Julia2012}:
\begin{equation}
B_{CU}(a) = U(a) - \lambda_{CU}C(a),
\label{equation:cost-utility}
\end{equation}

where $U(a)$ is a utility function, $C(a)$ is a cost function and $\lambda$ is a constant that adjusts the relative importance between both factors. Utility and cost functions are expressions normalized in the range $\left[0, 1\right]$ that are calculated as follows:
\begin{equation}
U(a) = \frac{U_{nex}(a, R_{s})}{\pi R_{s}^{2}},
\end{equation}
\begin{equation}
C(a) = \frac{L(a)}{max_{b \in F}L(b)},
\end{equation}

where the function $U_{nex}(a. R_{s})$ is the result of counting the number of unexplored cells in the range of the sensor from cell $d$, being $R_{s}$ the maximum range of the sensor expressed in cell units.
Then, the target cell $t_{CU}$ is chosen as the one that maximizes the utility-cost relation:

\begin{equation}
t_{CU} = \argmax_{a \in F} B_{CU}(a).
\end{equation}

Similar to the work proposed by Simmons, Burgard \cite{Burgard2000} introduced some coordination by means of reducing a determined initial utility given to each frontier depending on the likelihood of being in the sensor range from other frontiers that have been assigned to other robots. The assignment of frontiers to robots is made sequentially using a cost-utility approach with the length of the minimum path as cost. It is assumed that robots know each others relative positions. The algorithm determines optimal target points for each robot that increase the coverage by the maximum amount at that time period. Moreover, Burgard in \cite{Burgard2005} suggested that the assignment of frontiers to robots could be optimized using the Hungarian method \cite{Kuhn1955} instead of the sequential assignment.

Another example of cost-utility model is given in \cite{Umari2017}, where a frontier detection method is based on Rapidly Exploring Random Trees (RRTs). Umari defined revenue from a frontier point as a combination of an information gain and navigation cost. 
Bhattacharya et al. \cite{Bhattacharya2013}, \cite{BhattacharyaGhrist2013} focused on the use of cost functions related to information theory, casting the exploration problem as a minimization of map entropy. Authors combined this approach with a grid-based map decomposition with an entropy minimization which results in complete coverage of a known map and full exploration if the scenario is unknown. 

\subsection{Market-Based Approach}

The general concept of the market-based approaches includes independence of robots in terms of planning, and the ability of robots to take team resources into account.
Relatively close to Burgard's approach in \cite{Burgard2000}, Zlot \cite{Zlot2002} uses a market architecture for the multi-robot mapping and exploration problem that aims to minimize an overall exploration time. In market-based coordinated approach each robot contains a list of goal points and profits associated with (\ref{equation:cost-utility}). Each robot selects the most profitable target as destination and when after reaching a current goal point, a robot initiates an auction. For each point in auction, each robot makes a bid with its current profit aiming to minimize own travel distance and maximize new area information.

It is shown in \cite{Dias2003} when different team sizes are included, a market method has an advantage over a centralized approach in terms of traveled distance. 

Michael et al. \cite{Michael2008} proposed a marked-based coordination protocols where robots are able to bid for task assignment with the assumption that every robot has knowledge of the maximum number of robots that any given task can accommodate. Each auction is performed among neighboring groups of robots and requires only local communication.

Sheng et al. \cite{Sheng2006} proposed an algorithm based on a distributed bidding model to coordinate the movement of multiple robots with limited communication range. The bidding algorithm takes into consideration distances between robots and a map synchronization mechanism reduces the exchanged data volume when robot subnetworks merge.

\subsection{Decentralized Coordinated Approach}

Pereira et al. \cite{Pereira2015} proposed a new exploration and mapping strategy, which relies on individual decision rules and communication of topological maps to achieve efficient and fast mapping. In this distributed strategy each robot broadcasts a graph representing the topological map, which can be transmitted to robots that are not within the communication range (through other robots in a system). 

Decentralized coordinated approach described in \cite{Colares2016} showed how exploration efficiency can be greatly improved. Authors presented a decentralized approach for multi-robot exploration that leverages the classical frontier based methods. The strategy took into consideration a utility function (the information gain) and the distance costs of frontiers. In that manner, robots are able to coordinate themselves and avoid the exploration of redundant areas by exchanging information and merging maps.

Lopez-Perez et al. \cite{LopezPerez2018} proposed a new method to explore unknown areas by using a scene partitioning scheme and assigning weights to the frontiers between explored and unknown areas. Authors presented a distributed algorithm, which reduces the number of communications between robots as well as the time needed to explore unknown regions and the distance traveled by each robot. Algorithm implemented by Benkrid and Achour \cite{Benkrid2017} also assigns weights to frontier cells, but, on the oher hand, depends on the energy of the battery of each robot.

Another interesting approach is proposed by Faigl et al. \cite{Faigl2015} where a goal assignment is solved as task-allocation problem, where goal locations are repeatedly determined during the exploration. 

In the described approaches, the main focus is improving the next action in order to explore an unknown area as fast as possible. Authors aimed to reduce a distance traveled by the robots as well as information shared between robots. This work motivated us to implement decentralized strategy for autonomous multi-robot frontier exploration and mapping of an unknown area. In our approach a robot team simultaneously explores the environment, discovers frontier points and shares information in order to become dispersed throughout the environment. During the exploration, information exchanged between the robots is limited to data containing robot positions and current robot target points. The main goal of the approach is to allocate the mobile robots to target frontier points in a way which minimizes the overall exploration time. Moreover, a robot team at the same time creates a common map of the environment. Our approach is a hybrid one - the robots can independently decide towards which target point to navigate using an optimization procedure, while having common knowledge of all target frontier points and sharing information on their position and current goals.  our approach is hybrid and uses slightly different objective functions for frontier points assignment, which are a combination of frontier point cost, utility of reaching the target point and \textit{frontier occupancy function}. We also cluster frontier points to get a problem of manageable size and thus enable application of known optimization algorithms. 
Our approach is hybrid in a manner that target point assignment process and navigation are fully decentralized and event-based, that is, each robot team member makes an individual decision on the next target point each time it reaches the previous one. On the other side, SLAM extended with frontier detection and filter module are a centralized part of the exploration and mapping process.

Overview of the system is given in Fig. \ref{fig:exploration-strategy}. The system consists of a centralized part (Simultaneous Localization and Mapping (SLAM), frontier detection and frontier points filter), which runs on a dedicated computer and a decentralized part (Exploration strategy and navigation), which runs on each robot. 

\begin{figure}[t!]
	\centering\includegraphics[width=1.0\columnwidth]{./pictures/diagram_exploration.pdf}
	\caption{Overall schematic diagram of the decentralized exploration and mapping process for $n$ mobile robots in the simulator. Google Cartographer SLAM and filter module (highlighted in red) generate filtered frontier points that are (currently) the centralized part of exploration and mapping process. The exploration strategy, path planning and navigation module (highlighted green) are decentralized parts that generate $n$ outputs and create a common map.}
	\label{fig:exploration-strategy}
\end{figure}




