\section{Conclusion and future work} \label{sec:conclusion}

In this paper, a modular approach to autonomous decentralized multi-robot exploration and mapping was presented. This approach is not only restricted to Google Cartographer SLAM and dense frontier detection, but may also be applied to different multi-robot systems. This strategy has resulted in improved behaviour in terms of exploration time compared to a state-of-the-art strategy in terms of exploration time. 

Even though the goals of this paper were shown to be achieved, the algorithm is open towards improving. Future research should consider decentralized map creating. Also, a significant improvement to this strategy would be a simpler simulator, that would allow for simulation of more robots.

Another research direction can be an extension to the algorithm to cope with a limited communication range of the mobile robots. Future work should also consider a multi-robot system which uses a (not fully) connected communication graph. Finally, we would like to take into consideration scenarios in which the robots may fail as well as time-varying environment scenarios.

The survey presents two chapters covering important tasks
for multi-robot systems. Including motion planning and area
exploration should provide a wide insight into the current
leading edge. This work serves as a stepping stone for new
research in path planning or exploration algorithms.
The unsolved problems in motion planning, but also in the
field of exploration, present huge opportunity for research.
From the control perspective it must be mentioned that many
authors focus solely on planning paths, while the execution
of the given paths is disregarded. In that regard, adding a
sophisticated control algorithm to further improve trajectory
execution. The first step would be to use the framework based
in [55]. As the work focuses on a single mobile robot, a
full extension for multiple robots should be developed. As
the presented navigation function is continuous even using
an extension of the potential field theory as an repulsion
force between robots nearing collision or deadlock is being
evaluated