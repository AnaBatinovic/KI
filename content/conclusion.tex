\section{Conclusion and future work} \label{sec:conclusion}

The problem of autonomous environment exploration is complex, because it includes the control of a robot, an environment scanning with a 3D sensor, creating a volumetric consistent scene in a common coordinate system from multiple views, the computation of next view points to provide efficiently full coverage of the scene, an elimination of an 
occlusion under the constraint of minimizing the total
path length between these points. 

In this paper, a brief overview of the exploration strategies is given. 
%A modular approach to autonomous decentralized multi-robot exploration and mapping was presented. Even though the strategy achieved our expectations, the algorithm is open towards improving. Future research should consider decentralized map creating. Also, a significant improvement to this strategy would be a simpler simulator, that would allow for simulation of more robots.

Another research direction can be an extension to the 3D space including multi aerial vehicles (MAVs).  algorithm to cope with a limited communication range of the mobile robots. Future work should also consider a multi-robot system which uses a (not fully) connected communication graph. Finally, we would like to take into consideration scenarios in which the robots may fail as well as time-varying environment scenarios.

This paper presents a solution to autonomously explore 3D environment without requiring an initial map or goal point. Octomap itself is able to handle dynamic environments by updating the map which makes the
algorithm works well in dynamic environment as the point
cloud is extracted from it.

This work is a stepping stone for new research in multi-robot exploration algorithms and its decentralization. There is also huge opportunity in 3D mapping and exploration, since ...