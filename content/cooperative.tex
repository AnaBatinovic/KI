\section{Multi-robot system: Decision making}
8. Decision-making: Centralized Versus Decentralized
Decision-making can be regarded as a cognitive process
resulting in the selection of a course of action among
several alternative scenarios. Every decision-making
process produces a final choice. In MRS, the decisionmaking guided by planning can be centralized or
decentralized in accordance with the group architecture
of the robots.
There is a central control agent in centralized architectures
that has the global information about the environment as
well as all information about the robots, and which can
communicate with all the robots to share them. The central
control agent could be a computer or a robot. The
advantage of the centralized architecture is that the central
control agent has a global view of the world, whereby the
globally optimal plans can be produced. Nevertheless, this
architecture: 1) is typical for a small number of robots and
ineffectual for large teams with more robots; 2) is not
robust in relation to dynamic environments or failures in
communications and other uncertainties; 3) produces a
highly vulnerable system, and if the central control agent
malfunctions a new agent must be available or else the
entire team is disabled. A typical MRS using a centralized
architecture is GOFER [2], in which there is a central task
planning and scheduling system with a global view of the
tasks to be performed in the environment and the
availability of robots to perform the tasks. Certain studies
outlined in the previous sections belonging to the
centralized architecture approach include [66] [92] [7] [124]
[40] [71] [43] [49].
Decentralized architectures can be further divided into
two categories: distributed architectures and hierarchical
architectures. There is no central control agent in
distributed architectures, such that all the robots are
equal with respect to control and are completely
autonomous in the decision-making process. In
hierarchical architectures, there exist one or more local
central control agents which organize robots into clusters.
The hierarchical architecture is a hybrid architecture,
intermediate between a centralized architecture and a
distributed architecture. In contrast to a centralized
architecture, a decentralized architecture can better
respond to unknown or changing environments, and
usually has better reliability, flexibility, adaptability and
robustness. Nevertheless, the solutions they reach are
often suboptimal. Feddema et al. [146] have focused on
input / output reachability, structural observability,
system controllability and connection stability for
decentralized control systems. They also show how these
theories are applicable to multi-robot formation control,
perimeter surveillance and (surround and monitor an
enemy facility). A typical MRS using a distributed
architecture is M+ [5], in which each robot has its own
local knowledge of the world and can decide its future
actions by taking into account its current context and
task, its own capacities and the capacities of the other
robots, through a negotiation process. A typical MRS
using a hierarchical architecture is CEBOT [3], in which
cells (robots) can be physically coupled to others and
some master cells are selected to coordinate tasks’
execution. Certain studies outlined in the previous
sections belonging to the decentralized architecture
approach include [4] [121] [67] [63] [125] [29] [60] [26] [28]
[56] [32] [69] [44] [158] [161]