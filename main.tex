%%%%%%%%%%%%%%%%%%%%%%%%%%%%%%%%%%%%%%%%%%%%%%%%%%%%%%%%%%%%%%%%%%%%%%%%%%%%%%%%
%2345678901234567890123456789012345678901234567890123456789012345678901234567890
%        1         2         3         4         5         6         7         8

\documentclass[letterpaper, 10 pt, conference]{ieeeconf}  % Comment this line out
                                                          % if you need a4paper
%\documentclass[a4paper, 10pt, conference]{ieeeconf}      % Use this line for a4
                                                          % paper

\IEEEoverridecommandlockouts                              % This command is only
                                                          % needed if you want to
                                                          % use the \thanks command
\overrideIEEEmargins
% See the \addtolength command later in the file to balance the column lengths
% on the last page of the document

\pdfminorversion=4

% The following packages can be found on http:\\www.ctan.org
%\usepackage{graphics} % for pdf, bitmapped graphics files
%\usepackage{epsfig} % for postscript graphics files
%\usepackage{mathptmx} % assumes new font selection scheme installed
%\usepackage{times} % assumes new font selection scheme installed
%\usepackage{amsmath} % assumes amsmath package installed
%\usepackage{amssymb}  % assumes amsmath package installed
\usepackage{amsmath}    			% ams packages for mathematics environment    
\usepackage{amssymb}
\usepackage{amsfonts}
%\usepackage{amsthm}

\usepackage{graphicx}  				% Versatile graphics manipulation options

\usepackage[croatian]{babel}  % Croatian typographical rules and hyphenation patterns 
\usepackage[utf8]{inputenc}  	% Encoding of Croatian characters
\usepackage[T1]{fontenc}
\usepackage{ae,aecompl}     	% Type 1 fonts, similar to Computer Modern

\usepackage{microtype}				% Improves spacing

\usepackage{subfig}

\usepackage{tabularx}
\usepackage{booktabs}
%\newcolumntype{C}{>{\centering\arraybackslash}X} % centered version of "X" type
\setlength{\extrarowheight}{1pt}
\usepackage{enumerate}				% Additional options for listing of items in enumerate environment
\usepackage{algorithm2e}			% Writing pseudo-code
\usepackage{todonotes}				% Adding todo items
\usepackage{dirtree}					% Simple display of directory tree
\usepackage{hyperref}					% Managing cross-referencing


\usepackage{lmodern}
\usepackage{nccmath}
\usepackage{mathptmx}
\usepackage[keeplastbox]{flushend}
\usepackage{scalerel,stackengine}
\DeclareMathOperator*{\argmax}{argmax} % thin space, limits underneath in displays
\DeclareMathOperator*{\argmin}{argmin}
\stackMath
\newcommand\reallywidehat[1]{%
	\savestack{\tmpbox}{\stretchto{%
			\scaleto{%
				\scalerel*[\widthof{\ensuremath{#1}}]{\kern.1pt\mathchar"0362\kern.1pt}%
				{\rule{0ex}{\textheight}}%WIDTH-LIMITED CIRCUMFLEX
			}{\textheight}% 
		}{2.4ex}}%
	\stackon[-6.9pt]{#1}{\tmpbox}%
}
\parskip 1ex

%calligraphy packages
\usepackage{calrsfs}
\DeclareMathAlphabet{\pazocal}{OMS}{zplm}{m}{n}
\newcommand{\Ca}{\pazocal{C}}
\newcommand{\Oa}{\pazocal{O}}
\newcommand{\Va}{\pazocal{V}}
\newcommand{\Ua}{\pazocal{U}}
\newcommand{\Aa}{\pazocal{A}}
\newcommand{\Ta}{\pazocal{T}}
\newcommand{\La}{\pazocal{L}}

\newcommand{\Ja}{\pazocal{J}}

\graphicspath{{./figures/}}
\usepackage{float}

\providecommand{\indexterms}[1]{\textbf{\textit{Index terms---}} #1}

\title{\LARGE \bf
	A Survey on Multi-Robot Autonomous Coordinated Exploration Strategies
}
\author{Ana Batinovi\'{c} \\
	University of Zagreb, Faculty of Electrical Engineering and Computing \\
	Department of Control and Computer Engineering\\
	Laboratory for Robotics and Intelligent Control Systems (LARICS) \\
	Unska 3, 10000 Zagreb \\
	Email: ana.batinovic@fer.hr
}

\makeatletter
\newcommand{\removelatexerror}{\let\@latex@error\@gobble}
\newcommand{\mb}[1]{\boldsymbol{#1}}
\newcommand{\norm}[1]{\left\lVert#1\right\rVert}

\makeatother

\begin{document}
\maketitle

\thispagestyle{empty}
\pagestyle{empty}


%%%%%%%%%%%%%%%%%%%%%%%%%%%%%%%%%%%%%%%%%%%%%%%%%%%%%%%%%%%%%%%%%%%%%%%%%%%%%%%%
\begin{abstract}

This paper proposes a short overview of multi-robot exploration strategies of an unknown environment. The main idea of the exploration strategy is an appropriate region selection as well as the minimization of a cost function involving the distance traveled by the robots, the time it takes for them to finish the exploration, and others. The paper focuses on multi-robot systems that could accomplish an unknown area exploration under required conditions.
These strategies play an important role in 2D as well as in 3D spaces. Therefore, this paper presents a systematic survey on the existing literature on coordination and exploration in the both spaces. The paper describes our recent work in autonomous multi-robot exploration systems and mapping using decentralized approach.
A brief conclusion and further research perspectives are given at the end of the paper. 

 
\indexterms{Multi-Robot System, Target Point, Exploration, Mapping, Frontiers, OctoMap.}

\end{abstract}
\section{introduction}


Application of multi-robot systems to solving core robotics problems has drawn significant attention in the last few decades. One example is coordination of a robot team for exploration of an unknown area, which is encountered in many applications, such as search and rescue (\cite{Murphy2004}), cleaning (\cite{Endres}), (\cite{Pinheiro2015}), warehousing (\cite{Wurman2008}) or planetary exploration (\cite{Mataric2001}), to name a few. Due to the fact that autonomous multi-robot systems are entering society and as such will interact with people on a daily basis, development of efficient coordination algorithms becomes necessary.

Robots need a map in order to operate in a particular environment. The ability of robots to autonomously travel around an unknown environment gathering the necessary information to obtain a useful map for navigation is called autonomous exploration \cite{Julia2012}. 

Like in the human society, robots can be more effective when they work together. Moreover, a robot team can accomplish a predefined task much quicker than a single robot can (\cite{Dias2000}). Another advantage of robot teams is the possibility of sensor fusion, which in turn can help to compensate for sensor uncertainty (\cite{Wurm2008}).
If done properly, multi-robot coordination can lead to i) task accomplishment in shorter time, ii) increased robustness, iii) higher map quality, and finally iv) the completion of tasks impossible to be performed by a single robot (\cite{Dias2006}).

This paper focuses on different exploration strategies using multi-robot system with coordinated robots that could accomplish a given task in a way which minimizes the overall exploration time. These strategies play an important role in 2D as well as in 3D spaces. Therefore, this paper makes an extensive overview of the exploration strategies in the both spaces.
 
The rest of the paper is organized as follows. 
In Section II an overview of 2D exploration strategies is given. Section
III briefly describes existing 3D exploration strategies, and in the final section a conclusion is given.

\begin{figure}[t!]
	\centering
	\fbox{\includegraphics[width=1.0\columnwidth]{./pictures/rviz_environment.pdf}}
	\caption {The environment is represented by a 2D map, with an occupancy grid that divides the map into cells: white cells describe free while grey cells unknown space. Black cells define occupied space (obstacles). Frontiers (red) and filtered frontiers (green).}
	\label{fig:environment}
\end{figure}

%\begin{figure}
%	\centering
%	\includegraphics[width=0.85\columnwidth]{./pictures/manip_traj_omega.png}	
%	\caption{Control scheme for the UAV carrying a payload. When considering UAV with MMC \textit{Inverse Jacobian} block becomes the identity matrix. This means that the control inputs $d_x$ and $d_y$ are directly sent as system inputs.}
%	\label{fig:control_scheme}
%\end{figure}

%Zadaci: 
%1. abstract
%2. introduction
%3. 2d exploration - subsection jedan o mom - slika
%3. 3d exploration - subsection jedan o fbet - slika
%4. conclusion
\section{2d exploration strategies}

Exploration algorithms can be grouped into centralized and decentralized. In the  centralized approach, each mobile robot receives tasks from a single central \emph{leader} which runs the overall planning algorithm, and afterwards the mobile robot sends its info back to the leader. Centralized assignment may be less practical due to communication limits (\cite{Dias2000}), robustness issues (\cite{Dias2006}), or time required for algorithm execution and scalability (\cite{Julia2012}). An advantage of centralized approach is that optimal plans can be found (\cite{Yan2011}). 

In contrast to centralized approaches, in a decentralized approach, the mobile robots are completely independent throughout the exploration process. Each mobile robot has its own local knowledge of the world and can decide its future actions by taking into account its current context and tasks, its own capacities and the capacities of the other mobile robots, through a negotiation process (\cite{Yan2013}). Moreover, it typically has better reliability, flexibility, adaptability and robustness (\cite{Zlot2002}). 
 
There are several representative approaches from the both centralized and decentralized groups described in the following text. 



\subsection{Nearest Frontier Approach} 

\begin{figure}[t!]
	\centering
	\fbox{\includegraphics[width=1.0\columnwidth]{./pictures/rviz_environment.pdf}}
	\caption {The environment is represented by a 2D map, with an occupancy grid that divides the map into cells: white cells describe free while grey cells unknown space. Black cells define occupied space (obstacles). Frontiers (red) and filtered frontiers (green).}
	\label{fig:environment}
\end{figure}
An unexplored area is usually represented using an occupancy grid map introduced by \cite{Moravec}. While a robot moves, an occupancy likelihood for each cell of the grid is updated with the information of sensors. Depending on this occupancy likelihood, cells can be classified as free, occupied or unknown (Fig. \ref{fig:environment}). Using an occupancy grid a mobile robot can reach an unexplored zone navigating to the frontier cells that separate the free cells from the unknown cells known as \textit{frontiers} \cite{Yamauchi1997}. Yamauchi's technique consists in selecting the shortest path to the nearest frontier. In this way, the target cell selected by this technique $t_{NF}$ is:

\begin{equation}
t_{NF} = \argmin_{a \in F} L(a), 
\label{equation:t-nf}
\end{equation}

where $L(a)$ represents the length of the shortest path to reach the cell $a$ ($a_{i}$, $a_{j}$) and $F$ the subset of the frontier cells \cite{Julia2012}. As it can be noticed, (\ref{equation:t-nf}) only takes into account the cost of reaching a frontier cell and does not provide any coordination mechanism. In case a single mobile robot system is extended to a multi-robot system, mobile robots may select the same frontier if they are situated in nearby positions. For instance, \cite{Yamauchi1998} extended his nearest frontier approach to multiple mobile robots using global maps built by each robot with the information provided by all robots. Since mobile robots share the acquired information, exploration is cooperative, but mobile robot movements are uncoordinated. When the robots are in close positions it is likely that they choose the same frontier to explore if no other coordination mechanisms are considered.

Frontier exploration strategies are also extended to a multi-robot system in \cite{Simmons2000} and \cite{Burgard2005}. Simmons \cite{Simmons2000} used a semi-distributed model where a centralized module integrated local data from a team of mobile robots. The team used a probabilistic technique to build a global map in a coordinated fashion. Due to the problem of absolute positioning techniques in indoor environments, robots must estimate their local pose in an environment, leading to odometry errors. Simmons used probability calculations to estimate the local pose of each robot, and built a global map by joining each individual robots local map. 
While Simmons \cite{Simmons2000} implements coordination between robots by sharing of map information and reducing the utility of frontier points in the vicinity of an allotted point, Burgard \cite{Burgard2005} came up with an elegant bidding process. 

A dense frontier points detection method implemented by Orsulic (\cite{Orsulic2019}) is an extension to Google Cartographer (\cite{Hess2016}) that has achieved good results in terms of wall-time per frontier update, which greatly speeds up exploration process. Orsulic used nearest frontier approach in order to explore an area. 

Rekeleitis \cite{Rekeleitis2000} covered terrains with multiple robots where at least one robot was stationary and posed as an observer. In \cite{Fox2006} a decision theoretic approach to multi-robot exploration was presented where the main problem was to decide whether a mobile robot should explore the terrain or to verify the hypothesis of other robots whose states are not mapped into a common reference frame. The nearest unexplored region technique is also used by Wullschleger \cite{Wullschleger99}, Santosh \cite{Santosh2008}, Murphy and Newman \cite{Murphy2008}, to name a few more.


\subsection{Cost-Utility Approach}

Generally, in a cost-utility approach a goal point maximizes the benefit between cost and utility. The utility is measured in terms of the expectation of the information incorporated to the occupancy map from the position of the goal point. An example of cost-utility approach was presented by González-Baños and Latombe \cite{GonzlezBaos2002}. Frontier cells are designated as candidate destinations and the benefit $B_{CU}(a)$ to reach a candidate cell a is evaluated according to the following expression \cite{Julia2012}:
\begin{equation}
B_{CU}(a) = U(a) - \lambda_{CU}C(a),
\label{equation:cost-utility}
\end{equation}

where $U(a)$ is a utility function, $C(a)$ is a cost function and $\lambda$ is a constant that adjusts the relative importance between both factors. Utility and cost functions are expressions normalized in the range $\left[0, 1\right]$ that are calculated as follows:
\begin{equation}
U(a) = \frac{U_{nex}(a, R_{s})}{\pi R_{s}^{2}},
\end{equation}
\begin{equation}
C(a) = \frac{L(a)}{max_{b \in F}L(b)},
\end{equation}

where the function $U_{nex}(a. R_{s})$ is the result of counting the number of unexplored cells in the range of the sensor from cell $d$, being $R_{s}$ the maximum range of the sensor expressed in cell units.
Then, the target cell $t_{CU}$ is chosen as the one that maximizes the utility-cost relation:

\begin{equation}
t_{CU} = \argmax_{a \in F} B_{CU}(a).
\end{equation}

Similar to the work proposed by Simmons, Burgard \cite{Burgard2000} introduced some coordination by means of reducing a determined initial utility given to each frontier depending on the likelihood of being in the sensor range from other frontiers that have been assigned to other robots. The assignment of frontiers to robots is made sequentially using a cost-utility approach with the length of the minimum path as cost. It is assumed that robots know each others relative positions. The algorithm determines optimal target points for each robot that increase the coverage by the maximum amount at that time period. Moreover, Burgard in \cite{Burgard2005} suggested that the assignment of frontiers to robots could be optimized using the Hungarian method \cite{Kuhn1955} instead of the sequential assignment.

Another example of cost-utility model is given in \cite{Umari2017}, where a frontier detection method is based on Rapidly Exploring Random Trees (RRTs). Umari defined revenue from a frontier point as a combination of an information gain and navigation cost. 


\subsection{Market-Based Coordinated Approach}

The general concept of the market-based approaches includes independence of robots in terms of planning, and the ability of robots to take team resources into account.
Relatively close to Burgard's approach in \cite{Burgard2000}, Zlot \cite{Zlot2002} uses a market architecture for the multi-robot mapping and exploration problem that aims to minimize an overall exploration time. In market-based coordinated approach each robot contains a list of goal points and profits associated with (\ref{equation:cost-utility}). Each robot selects the most profitable target as destination and when after reaching a current goal point, a robot initiates an auction. For each point in auction, each robot makes a bid with its current profit aiming to minimize own travel distance and maximize new area information.

It is shown in \cite{Dias2003} when different team sizes are included, a market method has an advantage over a centralized approach in terms of traveled distance. Authors in \cite{Zlot2002} 

Michael et al. \cite{Michael2008} proposed a marked-based coordination protocols where robots are able to bid for task assignment with the assumption that every robot has knowledge of the maximum number of robots that any given task can accommodate. Each auction is performed among neighboring groups of robots and requires only local communication.

\subsection{Decentralized Coordinated Approach}

Exploration strategies include algorithms for assigning robots to target points for environment exploration. Such algorithms can be grouped into centralized and decentralized algorithms. In the  centralized approach, each mobile robot receives tasks from a single central \emph{leader} which runs the overall planning algorithm, and afterwards the mobile robot sends its info back to the leader. Centralized assignment may be less practical due to communication limits \cite{Dias2000}, robustness issues \cite{Dias2006}, or time required for algorithm execution and scalability \cite{Julia2012}. An advantage of centralized approach is that optimal plans can be found in\cite{Yan2011}. 

In contrast to centralized approaches, in a decentralized approach, the mobile robots are completely independent throughout the exploration process. Each mobile robot has its own local knowledge of the world and can decide its future actions by taking into account its current context and tasks, its own capacities and the capacities of the other mobile robots, through a negotiation process (\cite{Yan2013}). Moreover, it typically has better reliability, flexibility, adaptability and robustness (\cite{Zlot2002}). 

Our approach is a hybrid one - the robots can independently decide towards which target point to navigate using an optimization procedure, while having common knowledge of all target frontier points and sharing information on their position and current goals.  our approach is hybrid and uses slightly different objective functions for frontier points assignment, which are a combination of frontier point cost, utility of reaching the target point and \textit{frontier occupancy function}. We also cluster frontier points to get a problem of manageable size and thus enable application of known optimization algorithms. 
Our approach is hybrid in a manner that target point assignment process and navigation are fully decentralized and event-based, that is, each robot team member makes an individual decision on the next target point each time it reaches the previous one. On the other side, SLAM extended with frontier detection and filter module are a centralized part of the exploration and mapping process.

Overview of the system is given in Fig. \ref{fig:exploration-strategy}. The system consists of a centralized part (Simultaneous Localization and Mapping (SLAM), frontier detection and frontier points filter), which runs on a dedicated computer and a decentralized part (Exploration strategy and navigation), which runs on each robot. In the following text each part is described separately.

\begin{figure}[t!]
	\centering\includegraphics[width=1.0\columnwidth]{./pictures/diagram_exploration.pdf}
	\caption{Overall schematic diagram of the decentralized exploration and mapping process for $n$ mobile robots in the simulator. Google Cartographer SLAM and filter module (highlighted in red) generate filtered frontier points that are (currently) the centralized part of exploration and mapping process. The exploration strategy, path planning and navigation module (highlighted green) are decentralized parts that generate $n$ outputs and create a common map.}
	\label{fig:exploration-strategy}
\end{figure}
\section{3D exploration strategies}
--3D(Nearest frontier)
%In contrast the approaches for exploration in 3D terrains are relatively lesser. The notion of coverage of known terrains rather than exploration of the unknown has gained more prominence in 3D worlds. In [9] Eidenbenz presents an approach for minimum number of guards to cover a known 3D terrain. Recently Joho
%and others present a paper for single robot exploration in 3D
%worlds [10] through a metric that is analogous to . This and
%earlier approaches such as [11] are more focused on generating
%3D maps and deal with single robots operating on multiple
%planes. The degree of freedom along the height axis is almost
%zero as they are ground robots. Height variations occur in those
%approaches only due to undulations of the ground plane. Our
%survey showed no work based on 3D heightmap based multirobot exploration.


%[9] S Eidenbenz, “Approximation Algorithms for Terrain
%Guarding”, Information Processing Letters,82(2002): 99-105
%[10] Dominik Joho, Cyrill Stachniss, Patrick Pfaff, and Wolfram
%Burgard, “Autonomous Exploration for 3D Map Learning”,
%Autonome Mobile Systeme (AMS). Kaiserslautern, Germany,
%2007.
%[11] Surmann H, Nuchter A, Hertzberg J, “An autonomous
%mobile robot with a 3D laser range finder for 3D exploration and
%digitalization of indoor environments”, Robotics and
%Autonomous Systems, 45(3-4):181–198, 2003. 

\subsection{Non-coordinated strategies}
\begin{figure}
	\centering
	\includegraphics[width=1.0\columnwidth]{./pictures/rviz_gazebo.pdf}	
	\caption{}
	\label{fig:rviz_gazebo}
\end{figure}
\section{Conclusion and future work} \label{sec:conclusion}

The problem of autonomous environment exploration is complex, because it includes the control of a robot, an environment scanning with a 3D sensor, creating a volumetric consistent scene in a common coordinate system from multiple views, the computation of next view points to provide efficiently full coverage of the scene, an elimination of an 
occlusion under the constraint of minimizing the total
path length between these points. 

In this paper, a brief overview of the exploration strategies is given. 
A modular approach to autonomous decentralized multi-robot exploration and mapping was presented. Even though the strategy achieved our expectations, the algorithm is open towards improving. Future research should consider decentralized map creating. Also, a significant improvement to this strategy would be a simpler simulator, that would allow for simulation of more robots.

Another research direction can be an extension to the 3D space including multi aerial vehicles (MAVs).  algorithm to cope with a limited communication range of the mobile robots. Future work should also consider a multi-robot system which uses a (not fully) connected communication graph. Finally, we would like to take into consideration scenarios in which the robots may fail as well as time-varying environment scenarios.

Autonomous area exploration has become a salient research
area in the field of robotics due to its various number of
applications. To handle the real world scenarios, the system
should be able to work in 3D environments which can be either
discovered or undiscovered previously. This research paper
presents a solution to autonomously explore 3D environments
without requiring an initial map or goal point. It builds a map
of the discovered area while exploring and extracts details of
the map like obstacles, planar surfaces and inclined surfaces.
Considering the relative inclination of surfaces, traversability
decisions are made by taking the specifications of the robot
to concern. The developed system has been successfully . Octomap itself is able to handle
dynamic environments by updating the map which makes the
algorithm works well in dynamic environment as the point
cloud is extracted from it.

T


T This work serves as a stepping stone for new
research in path planning or exploration algorithms.
The unsolved problems in motion planning, but also in the
field of exploration, present huge opportunity for research.
 As the work focuses on a single mobile robot, a
full extension for multiple robots should be developed.

%%%%%%%%%%%%%%%%%%%%%%%%%%%%%%%%%%%%%%%%%%%%%%%%%%%%%%%%%%%%%%%%%%%%%%%%%%%%%%%%
%\section*{APPENDIX} \label{sec:appendix}
%\input{content/appendix.tex}

%Appendixes should appear before the acknowledgment.

%\section*{ACKNOWLEDGMENT}



%%%%%%%%%%%%%%%%%%%%%%%%%%%%%%%%%%%%%%%%%%%%%%%%%%%%%%%%%%%%%%%%%%%%%%%%%%%%%%%%

%\nocite{*}
\bibliographystyle{ieeetr}

\bibliography{bibliography/Mendeley}

\end{document}
